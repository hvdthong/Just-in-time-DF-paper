\section{Related Work}
\label{sec:related}

In this section, we situate our work within the context of past studies on JIT modelling and its assumptions.

\subsection{JIT Defect Modelling}

Recent work has shown that JIT models have become sufficiently robust to be applied in practice.
Indeed, JIT models are included in the development processes of large software systems at Avaya~\cite{mockus2000bell}, Blackberry~\cite{shihab2012fse}, and Cisco~\cite{tan2015seip}.

To achieve such robustness, a variety of code change properties need to be used to predict fix-inducing changes.
For example, Mockus and Weiss~\cite{mockus2000bell} predicted fix-inducing changes using a set of code change properties that are primarily derived from the changes themselves.
Kim \ea~\cite{kim2008tse} and Kamei \ea~\cite{kamei2013tse} built upon the set of code change properties, reporting that the addition of a variety of properties that were extracted from the Version Control System (VCS) and the Issue Tracking System (ITS) helped to improve the prediction accuracy.
Kononenko \ea~\cite{kononenko2015icsme} also found that the addition of code change properties that were extracted from code review databases contributed a significant amount of explanatory power to JIT models.

We derive the majority of our set of studied code change properties from those described in these prior studies.
However, while these previous studies empirically evaluate the prediction performance of JIT models, they do not focus on the consistency of the properties of fix-inducing changes, which is the central thrust of this paper.

\subsection{Assumptions of Defect Modelling}
\label{sec:related_bias}

\begin{table*}
\centering
\caption{An overview of the studied systems. Those above the double line satisfy our criteria for analysis.}
\label{tab:cases}
\resizebox{\textwidth}{!}{%
\begin{tabular}{c||c|c||c|c||c|c|c}
\hline
\multirow{2}{*}{\bf System} & \multicolumn{2}{c||}{\bf Timespan} & \multicolumn{2}{c||}{\bf Changes} & \multicolumn{3}{c}{\bf Reviews} \\
\cline{2-8}
& {\bf Start} & {\bf End} & {\bf Total} & {\bf Defective} & {\bf Reviewed} & {\bf Self reviewed} & {\bf \# Reviewers} \\
\hline
{\sc Qt} & 06/2011 & 03/2014 & 25,150 & 2,002 (8\%) & 23,821 (95\%) & 1,217 (5\%) & $\bar{x} = 2.3$, $M = 1.0$, $\textit{sd} = 2.8$ \\
\hline
{\sc OpenStack} & 11/2011 & 02/2014 & 12,374 & 1,616 (13\%) & 12,041 (97\%) & 117 ($<$1\%) & $\bar{x} = 4.3$, $M = 3.0$, $\textit{sd} = 5.3$\\
\hline \hline
{\sc ITK} & 08/2010 & 08/2014 & \multicolumn{1}{r|}{3,347} & -  & -  & -  & -\\ \hdashline
{\sc VTK} & 08/2010 & 08/2014 & \multicolumn{1}{r|}{7,723} & -  & \multicolumn{1}{r|}{4,237 (55\%)} & - & -\\
{\sc Android} & 08/2010 & 01/2013 & 61,789 & - & \multicolumn{1}{r|}{2,771 (4\%)} & - & -\\
{\sc LibreOffice} & 03/2012 & 11/2013  & 11,988 & - & \multicolumn{1}{r|}{1,679 (14\%)} & - & -\\
\hline
\end{tabular}
}
\end{table*}


Past work has shown that if care is not taken when collecting data from software repositories, noise may impact defect models.
Aranda and Venolia~\cite{aranda2009icse} found that VCSs and ITSs are noisy sources of data.
For example, Antoniol \ea~\cite{antoniol2008cascon} found that issue reports are often mislabelled, i.e., reports that describe defects are labelled as enhancements, and vice versa.
Herzig \ea~\cite{herzig2013icse} found that this issue report mislabelling can impact the ranking of modules that is produced by defect models.
Furthermore, Bird \ea~\cite{bird2009fse} found that characteristics of issue reports (e.g., severity) and issue reporters (e.g., experience) can influence the likelihood of linking related ITS and VCS records when it is necessary.
Without these links, datasets that are constructed for defect prediction purposes may erroneously label defective modules as clean, or vice versa.
To study the impact that missing links may have on defect prediction, Kim \ea~\cite{kim2011icse} sever existing links to simulate a variety of rates of missing links.

On the other hand, recent work also shows that noise and bias may not be a severe impediment to defect modelling.
For example, our prior work~\cite{tantithamthavorn2015icse} showed that issue report mislabelling rarely impacts the precision of defect models or the interpretation of the top-ranked variables.
Nguyen \ea~\cite{nguyen2010wcre} found that biases exist even in a ``near-ideal'' development setting, where links are automatically recorded, suggesting that linkage bias is a symptom of any development process.
Moreover, Rahman \ea~\cite{rahman2013fse} found that the total number of modules has a larger impact on defect model performance than noise or biases do.

Even with data that is completely free of noise and bias, there are assumptions that must be satisfied in order to fit defect models.
Turhan~\cite{turhan2012emse} argued that ``dataset shifts'' (i.e., dataset characteristics that violate modelling assumptions~\cite{menzies2013tse}) may influence predictive models in software engineering.
In a similar vein, we study whether properties of fix-inducing changes are consistent enough to satisfy the assumption that past events are similar to future events.

Past studies also use historical data to improve the performance of defect prediction models~\cite{kim2007icse, nam2013icse,zimmermann2008evolution}.
For example, Nam \ea~\cite{nam2013icse} mitigated dataset shifts by applying a transfer learning approach, which makes feature distributions in training and testing datasets more similar.
Zimmermann \ea~\cite{zimmermann2008evolution} showed how complexity, problem domain, and change rate can be used to predict defects in Microsoft and Eclipse systems.
While these studies predict which modules are at-risk of containing defects, our study focuses on understanding the impact that various code change properties have on the risk that changes pose for inducing future fixes in a longitudinal setting.

Perhaps the most similar prior work is that of Ekanayake \ea~\cite{ekanayake2009msr}, who studied the stability of module-level defect prediction models.
They used trends in the predictive power of defect prediction models to identify periods of ``concept drift,'' i.e., periods where historical trends do not aid in identifying defect-prone modules. 
Our work differs from that of Ekanayake \ea~in two ways.
First, we study JIT models, which are concerned with fix-inducing changes rather than defective modules.
Second, while past work focuses on model performance, we study the impact that fluctuations in the importance of properties of fix-inducing changes have on both the performance (RQ1) and the interpretation (RQ2, RQ3) of JIT models.
