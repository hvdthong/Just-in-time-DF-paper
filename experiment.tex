\section{Experiments}
\label{sec:exp}
In this section, we first describe our dataset and how we preprocess it. We
then introduce all baselines and evaluation metrics. Finally, we present
our research questions and results.

\subsection{Dataset}
\subsection{Baseline}
\label{sec:baseline}
We compare our proposed model with the Just-in-Time (JIT) defect prediction approach mentioned in~\cite{mcintosh2018fix}. Different from the JIT model used to identify fix-inducing code changes, we train the model to identify bugs. We also use a nonlinear variant of multiple regression modeling to fit the JIT model. The nonlinear regression modeling has widely used in software engineering to understand the relationship between software development practices and software quality~\cite{zhou2011does, morales2015code, mcintosh2016empirical}. 

\subsection{Evaluation Metric and Setting}
\label{sec:metric_setting}

\subsection{Research Questions and Results}
\label{sec:rq_results}

\noindent \textbf{RQ1: How effective is our proposed deep learning model compared to the state-of-the-art baseline?}

\begin{figure}
\center
\includegraphics[scale=0.36]{figs/split.pdf}
\caption{An example of choosing the data for training proposed model. The last period will be used as testing data.}
\label{fig:splitting}
\end{figure}

\begin{table*}[t!]
  \centering
  \caption{The AUC results of DeepJIT vs. with other baselines in three settings: short-period, long-period, and random}
    \begin{tabular}{|l|c|c|c|c|c|c|}
    \hline
    \multirow{2}[4]{*}{} & \multicolumn{3}{c|}{QT} & \multicolumn{3}{c|}{OPENSTACK} \\
\cline{2-7}          & \multicolumn{1}{l|}{Short-Period} & \multicolumn{1}{l|}{Long-period} & \multicolumn{1}{l|}{Random} & \multicolumn{1}{l|}{Short-Period} & \multicolumn{1}{l|}{Long-period} & \multicolumn{1}{l|}{Random} \\
    \hline
    \hline
    JIT   & 0.703 & 0.702 & 0.701 & 0.711 & 0.706 & 0.691 \\
    \hline
    DBNJIT & 0.714 & 0.708 & 0.705 & 0.716 & 0.712 & 0.694 \\
    \hline
    DeepJIT & \textbf{0.764} & \textbf{0.765} & \textbf{0.768} & \textbf{0.781} & \textbf{0.771} & \textbf{0.751} \\
    \hline
    \end{tabular}%
  \label{tab:results}%
\end{table*}%

\noindent \textbf{RQ2: Does the proposed model benefit both commit message and the code changes?}

\begin{table*}[t!]
  \centering
  \caption{Contribution of feature components in DeepJIT}
    \begin{tabular}{|l|c|c|c|c|c|c|}
    \hline
    \multirow{2}[4]{*}{} & \multicolumn{3}{c|}{QT} & \multicolumn{3}{c|}{OPENSTACK} \\
\cline{2-7}          & \multicolumn{1}{l|}{Short-Period} & \multicolumn{1}{l|}{Long-period} & \multicolumn{1}{l|}{Random} & \multicolumn{1}{l|}{Short-Period} & \multicolumn{1}{l|}{Long-period} & \multicolumn{1}{l|}{Random} \\
    \hline
    \hline
    DeepJIT-Msg & 0.609 & 0.638 & 0.641 & 0.583 & 0.659 & 0.689 \\
    \hline
    DeepJIT-Code & 0.734 & 0.727 & 0.738 & 0.769 & 0.738 & 0.729 \\
    \hline
    DeepJIT & \textbf{0.764} & \textbf{0.765} & \textbf{0.768} & \textbf{0.781} & \textbf{0.771} \\
    \hline
    \end{tabular}%
  \label{tab:variants}%
\end{table*}%

\noindent \textbf{RQ3: Does the proposed model benefit from the manually extracted code changes features?}

\begin{table*}[t!]
  \centering
  \caption{Combination of DeepJIT with JIT's features}
    \begin{tabular}{|l|c|c|c|c|c|c|}
    \hline
    \multirow{2}[4]{*}{} & \multicolumn{3}{c|}{QT} & \multicolumn{3}{c|}{OPENSTACK} \\
\cline{2-7}          & Short-Period & Long-period & Random & Short-Period & Long-period & Random \\
    \hline
    \hline
    DeepJIT & 0.764 & 0.765 & 0.768 & 0.781 & 0.771 & 0.751 \\
    \hline
    DeepJIT-Combined & \textbf{0.788} & \textbf{0.786} & \textbf{0.779} & \textbf{0.814} & \textbf{0.799} & \text{0.760} \\
    \hline
    \end{tabular}%
  \label{tab:combined}%
\end{table*}%

\noindent \textbf{RQ4: How are the time costs of the proposed model?}


