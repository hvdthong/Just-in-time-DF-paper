\documentclass[conference]{IEEEtran}
\IEEEoverridecommandlockouts
% The preceding line is only needed to identify funding in the first footnote. If that is unneeded, please comment it out.
\usepackage{cite}
\usepackage{amsmath,amssymb,amsfonts}
\usepackage{algorithmic}
\usepackage{graphicx}
\usepackage{textcomp}
\usepackage{xcolor}
\usepackage{booktabs} % For formal tables
\usepackage{multirow}
\usepackage{hyperref}
\usepackage{subcaption}
\def\BibTeX{{\rm B\kern-.05em{\sc i\kern-.025em b}\kern-.08em
    T\kern-.1667em\lower.7ex\hbox{E}\kern-.125emX}}

\newcommand{\cmt}[1]{\textbf{\textcolor{red}{#1}}}

\begin{document}

\title{DeepJIT: An \emph{End-To-End} Deep Learning Framework for \emph{Just-In-Time} Defect Prediction}
%
%\author{\IEEEauthorblockN{1\textsuperscript{st} Given Name Surname}
%\IEEEauthorblockA{\textit{dept. name of organization (of Aff.)} \\
%\textit{name of organization (of Aff.)}\\
%City, Country \\
%email address}
%\and
%\IEEEauthorblockN{2\textsuperscript{nd} Given Name Surname}
%\IEEEauthorblockA{\textit{dept. name of organization (of Aff.)} \\
%\textit{name of organization (of Aff.)}\\
%City, Country \\
%email address}
%\and
%\IEEEauthorblockN{3\textsuperscript{rd} Given Name Surname}
%\IEEEauthorblockA{\textit{dept. name of organization (of Aff.)} \\
%\textit{name of organization (of Aff.)}\\
%City, Country \\
%email address}
%\and
%\IEEEauthorblockN{4\textsuperscript{th} Given Name Surname}
%\IEEEauthorblockA{\textit{dept. name of organization (of Aff.)} \\
%\textit{name of organization (of Aff.)}\\
%City, Country \\
%email address}
%\and
%\IEEEauthorblockN{5\textsuperscript{th} Given Name Surname}
%\IEEEauthorblockA{\textit{dept. name of organization (of Aff.)} \\
%\textit{name of organization (of Aff.)}\\
%City, Country \\
%email address}
%\and
%\IEEEauthorblockN{6\textsuperscript{th} Given Name Surname}
%\IEEEauthorblockA{\textit{dept. name of organization (of Aff.)} \\
%\textit{name of organization (of Aff.)}\\
%City, Country \\
%email address}
%}

\maketitle

\begin{abstract}
Software Quality Assurance efforts often focus on identifying software modules that are likely to be defective in the future. To solve the problem of finding likely defective code early, change-level defect prediction -- aka. \emph{Just-In-Time} (JIT) defect prediction -- has been proposed to identify buggy code changes. JIT models are trained using machine learning techniques which assume that historical changes are similar to future ones. Hence, these changes can be used to identify defect-prone software modules (e.g., functions, files, system, etc.). A previous approach relies on manually extracted code changes features. This approach, however, shows only moderate accuracy. In this paper, we propose an end-to-end deep learning framework, namely DeepJIT, that is automatically extracting features from commit messages and code changes and using them to identify bugs. Experiments on two popular software projects (i.e., QT and OPENSTACK) on three evaluation settings (i.e., cross-validation, short-period, and long-period) show that DeepJIT achieves improvements of 10.50\%, 10.36\%, and 11.02\% in the project QT and 9.51\%, 13.69\%, 12.22\% in the project OPENSTACK in terms of the Area Under the Curve (AUC). 
\end{abstract}
\section{Introduction}
\label{sec:intro}
As software systems are becoming the backbone of our economy and society, defects existing in those systems may substantially affect businesses and people's lives  in many ways. For example, Knight Capital\footnote{https://dealbook.nytimes.com/2012/08/02/knight-capital-says-trading-mishap-cost-it-440-million/}, a company which executes automated trading  for retail brokers, lost $\$$440 millions in only one morning in 2012 due to an overnight faulty update to its trading software. A flawed code change, introduced into OpenSSL's source code repository, caused the infamous Heartbleed\footnote{http://heartbleed.com} bug which affected billions of Internet users in 2014. As software grows significantly in both size and complexity, finding defects and fixing them become increasingly difficult and costly.

One common best practice for cost saving is identifying defects and fixing them as early as possible, ideally before new code changes (i.e. \emph{commits}) are introduced into codebases. Emerging research has thus developed \emph{Just-In-Time} (JIT) defect prediction models and techniques which help software engineers and testers to quickly narrow down the most likely defective commits to a software codebase~\cite{KameiS16,D'Ambros:2012:EDP}. JIT defect prediction tools provide early feedback to software developers to prioritize and optimize effort for inspection and (regression) testing, especially when facing with deadlines and limited resources. They have therefore been integrated into the development practice at large software organizations such as Avaya~\cite{Mockus2000}, Blackberry~\cite{Shihab:2012:ISR}, and Cisco~\cite{Tantithamthavorn:2015:IMP}.

Machine learning techniques have been widely used in existing work for building JIT defect prediction models. A common theme of existing work (e.g.~\cite{Kamei:2013:LES,Kim:2008:CSC,Kononenko:2015,Mockus2000}) is carefully crafting a set of features to represent a code change, and using them as defectiveness predictors. Those features are mostly derived from properties of code changes, such as change size (e.g. lines deleted or added), change scope (e.g. number of files or directories modified), history of changes (e.g. number of prior changes to the updated files), track record of the author and code reviewers, and activeness of the code review of the change. This set of features can then be used as an input to a traditional classifier (e.g. Random Forests or Logistic Regression) to predict the defectiveness of code changes. 

The aforementioned metric-based features however do not represent the semantic and syntactic structure of the actual code changes. In many cases, two different code changes which have exactly the same metrics (e.g. the number of lines deleted and added) may generate different behaviour when executed, and thus have a different likelihood of defectiveness. Previous studies have showed the usefulness of harvesting the semantic information and syntactic structure hidden in source code to perform various software engineering tasks such as code completion, bug detection and defect prediction~\cite{Wang:2016:ALS,Tu:2014:LS,Nguyen:2015:GSL,Hindle:2012:NS,Li:2005:PAE}. This information may enrich representations for defective code changes, and thus improve JIT defect prediction.

A recent work~\cite{Yang:2015:DLJ}  used a deep learning model (i.e. Deep Belief Network) to improve the performance of JIT defect prediction models. However, their approach does not leverage the true notions of deep learning as they still employ the same set of features that are manually engineered as in previous work, and their model is \emph{not} end-to-end trainable.

% In this paper, we present a new JIT defect prediction model (namely DeepJIT) which leverages the powerful deep learning Convolution Neural Network (CNN) architecture to learn a deep representation of commits. Our model processes both a commit message (in natural language) and the associated code changes (in programming languages) and automatically semantic features which represent the ``meaning'' of the commit. This approach removes software practitioners from manually designing and extracting features, as done in previous work. DeepJIT is a fully end-to-end trainable system where raw data signals (e.g. words or code tokens) are passed from input nodes up to the final output node for predicting defectiveness, and prediction errors are back propagated from the output node down to the input layer.

To more fully explore the power of deep learning for JIT defect prediction, in this paper, we present a new model (named DeepJIT) which is built upon the well-known deep learning technique, namely Convolutional Neural Network (CNN)~\cite{lecun2015deep}. CNN has produced many breakthroughs in Natural Language Processing (NLP)~\cite{kim2014convolutional, dos2014deep, kalchbrenner2014convolutional, zhang2015character, johnson2014effective}.   
Our DeepJIT model processes both a commit message (in natural language) and the associated code changes (in programming languages) and automatically extracts features which represent the ``meaning'' of the commit. Unlike commit messages, code changes are more complex as they include a number of deleted and added lines across multiple files. We first learn the semantic features of each deleted or added line for each changed file using CNN framework. The features of the deleted and added lines are then incorporated to generate a new representation of the changed file. We again use CNN to extract features from the new representation of the changed file. The features of all the changed files are then used to construct the features of the code changes in a given commit. This approach removes the need for software practitioners to manually design and extract features, as what was done in previous work~\cite{mcintosh2018fix}. The features extracted from commit messages and code changes are then collectively used to train a model, and then predict whether a given commit is buggy or not. 

% our deep learning \emph{Just-In-Time} (JIT) defect prediction model (DeepJIT) aims to automatically extract features from both code changes and commit messages. These features are then collectively used to train a model that can be used to evaluate whether a given commit is buggy or not.
% % The key challenge here is to capture the semantic and syntactical structure of the code changes in the given commit. 
% Different from the commit message, the code changes is more complex as it includes a number of deleted and added lines across multiple files (see Figure~\ref{fig:example}). To solve this problem, we first learn the semantic features of each deleted or added line for each changed file using CNN framework, the semantic features of the added and deleted lines are then incorporated to build a new representation 
% % preserving the syntactical structure 
% of the change file. We again use CNN to extract features from the new representation of the change file. The features of all the changed files are used to construct the features of the code changes in a given commit.  

The main contributions of our paper include:
\begin{itemize}
    \item An end-to-end deep learning framework (DeepJIT) to automatically extract features from both commit messages and code changes in a given commit. 
    % DeepJIT is able to learn the semantic features from commit message and capture the semantic and syntactical structure of the code changes. 
    % DeepJIT follows the syntactical structure of the code changes to extract 
    \item An evaluation of DeepJIT on two software projects (i.e., QT and OPENSTACK). This dataset was originally collected by McIntosh and Kamei to evaluate their proposed technique~\cite{mcintosh2018fix} that we use as one of the baselines. The  experiments show the  superiority of DeepJIT compared to state-of-the-art baselines.
\end{itemize}


\section{Motivation}
\label{sec:motivation}

\subsection{An example of a commit }
\label{sec:examle}

\subsection{Convolutional Neural Networks}
\label{sec:background_cnn}

\begin{figure*}[t!]
\center
\includegraphics[scale=0.3]{figs/cnn.pdf}
\caption{A simple convolutional neural network architecture.}
\label{fig:cnn}
\end{figure*}

One of the most powerful forms of deep learning neural networks is the Convolutional Neural Network (CNN)~\cite{lecun2015deep}. CNNs are widely used to solve image pattern recognition problems and have been achieved significant results~\cite{karpathy2014large, lawrence1997face, krizhevsky2012imagenet}. Like traditional deep learning networks, CNNs receive an input and perform a product operation followed by a nonlinear function. The last layer is the output layer containing objective functions~\cite{zhao2017loss} associated with the labels of the input.

Figure~\ref{fig:cnn} illustrates a simple CNN for classification task. The simple CNN includes an input layer, a convolutional layer, followed by the rectified linear unit (RELU) which is a nonlinear activation function, a pooling layer, a fully-connected layer, and an output layer in the following paragraphs. 

The input layer takes an input as 2-dimensional array or 3-dimensional array and passes it through a of convolution layers.

The convolutional layer plays a vital role in CNN and it takes advantage of the use of learnable kernels. These kernels are small in spatial dimensionality, but they are applied along the entirety of the depth of the input data. For example, given an input data $\textbf{I} \in \mathbb{R}^{\text{h} \times \text{w} \times \text{d}}$ and a filter $\textbf{F} \in $

The convolutional layer is used to determine the output of neurons which are connected to local regions of the input through the calculation of the scalar product between a filter and the regions of the input data. Specifically, if the dimension of an input data is $h \times w \times d$, given a filter $f_h \times f_w \times d$, we output a new dimension $(h - f_h + 1) \times (w - f_w + 1) \times 1$. The RELU, which is a nonlinear activation function, is then applied to the new dimension as follows: 
\begin{equation}
\label{eq:relu}
f(x) = max(0, x)   
\end{equation}





\section{approach}
\label{sec:approach}
In this section, we first formulate the just-in-time defect prediction and provide an overview of our framework. We then describe 
\subsection{Settings}


\section{Experiments}
\label{sec:exp}
In this section, we first describe the dataset used in our paper. We then introduce all baselines and the evaluation metric. Finally, we present our research questions and results.

\subsection{Dataset}
\label{sec:dataset}
We used two well-known software projects (i.e., QT and OPENSTACK) to evaluate the performance of \emph{Just-In-Time} (JIT) models. QT~\footnote{\url{https://www.qt.io/}}, developed by the Qt Company, is a cross-platform application framework and allows contributions from individual developers and organizations. On the other hand, OPENSTACK~\footnote{\url{https://www.openstack.org/}} is an open-source software platform for cloud computing and is deployed as an infrastructure-as-a-service which allows customers to access its resources. 

\begin{table}[ht!]
  \centering
  \caption{Summary of the dataset used in this work}
    \begin{tabular}{|c|c|c|c|c|}
    \hline
    \multirow{2}[4]{*}{\textbf{Dataset}} & \multicolumn{2}{c|}{\textbf{Timespan}} & \multicolumn{2}{c|}{\textbf{Commits}} \\
\cline{2-5}          & \textbf{Start} & \textbf{End} & \textbf{Total} & \textbf{Defective} \\
    \hline
    \hline
    QT    & 06/2011 &  03/2014 & 25,150 & 2,002 (8\%) \\
    \hline
    OPENSTACK & 11/2011 &  02/2014 & 12,374 & 1,616 (13\%) \\
    \hline
    \end{tabular}%
  \label{tab:data}%
\end{table}%

Table~\ref{tab:data} briefly summarizes the dataset used in our paper. This dataset was originally collected and cleaned by McIntosh and Kamei~\cite{mcintosh2018fix}. After their cleaning process, the QT dataset contains 25,150 commits, while the OPENSTACK dataset contains 12,374 commits. McIntosh and Kamei stratified the dataset into six months periods for time-sensitive training-and-testing settings.

\subsection{Baselines}
\label{sec:baseline}

\begin{table*}[ht!]
	\centering
	\caption{A summary of McIntosh and Kamei's code features~\cite{mcintosh2018fix}.}
	\label{tab:metrics}
	\resizebox{\textwidth}{!}{
		\begin{tabular}{|c|p{2cm}|p{5.7cm}|p{8.4cm}|}
			\hline
			& {\bf Property} & {\bf Description} & {\bf Rationale} \\
			\hline
			\hline
			\multirow{2}{*}{\rotatebox{90}{Size}}
			& Lines deleted  & The number of deleted lines. &  The more deleted or added code, the more likely that defects \\
			\cline{2-3}
			& Lines added & The number of added lines. & may appear~\cite{nagappan2006icse,Kamei2010}.\\
			\hline
			\multirow{4}{*}{\rotatebox{90}{Diffusion}}
			& Subsystems & The number of modified subsystems. & Scattered changes may have more defects compared to focused one~\cite{Ambros2010, HassanICSE09}.\\
			\cline{2-3}
			& Directories & The number of modified directories. & \\
			\cline{2-3}
			& Files & The number of modified files. &  \\
			\cline{2-3}
			& Entropy & The spread of modified lines across file. &  \\
			\hline
			\multirow{6}{*}{\rotatebox{90}{History}}
			& Unique changes & The number of prior changes to the modified files. & More changes may lead to have defects since developers need to track many previous changes~\cite{Kamei:2013:LES}.\\
			\cline{2-4}
			& Developers & The number of developers who have changed the modified files in the past. & Files touched by many developers may include defects~\cite{matsumoto2010promise}. \\
			\cline{2-4}
			& Age & The time interval between the last and current changes. & More recently changed code likely contains defects compared to older code~\cite{Graves2000}. \\
			\hline
			\multirow{11}{*}{\rotatebox{90}{Author/Rev. Experience}}
			& Prior changes & The number of prior changes that an actor has participated in. & Changes produced by novices are likely to be more defective than changes produced by experienced developers~\cite{Mockus2000}. \\
			\cline{2-3}
			& Recent changes & The number of prior changes that an actor has participated in weighted by the age of the changes (older changes are given less weight than recent ones). & \\
			\cline{2-3}
			& Subsystem changes & The number of prior changes to the modified subsystem(s) that an actor has participated in. & \\
			\cline{2-4}
			& Awareness & The proportion of the prior changes to the modified subsystem(s) that an actor has participated in. & Changes made by developers who are aware of the prior changes in the impacted subsystems are likely to be less risky. \\
			\hline
			\multirow{12}{*}{\rotatebox{90}{Review}}
			& Iterations & Number of times that a change was revised prior to integration. & The quality of a change likely improves with each iteration. Hence, changes that undergo iterations prior to integration may be less risky~\cite{porter1998tosem, thongtanunam2015msr}.\\
			\cline{2-4}
			& Reviewers & Number of reviewers who have voted on whether a change should be integrated or abandoned. & Changes observed by many reviewers are likely to be less risky~\cite{Raymond2001}. \\
			\cline{2-4}
			& Comments & The number of non-automated, non-owner comments posted during the review of a change. & Changes with short discussions may be more risky~\cite{mcintosh2014impact, mcintosh2016empirical}.\\
			\cline{2-4}
			& Review window & The length of time between the creation of a review request and its final approval for integration. & Changes with shorter review windows may be more risky~\cite{porter1998tosem, thongtanunam2015msr}.\\
			\hline
		\end{tabular}
	}
\end{table*}

We compared DeepJIT with two state-of-the-art baselines for \emph{Just-In-Time} (JIT) defect prediction:

\begin{itemize}
\item JIT: This method for identifying buggy code changes was proposed by McIntosh and Kamei~\cite{mcintosh2018fix}. The method used a nonlinear variant of multiple regression modeling~\cite{fox1997applied} to build a classification model for automatically identifying defects in commits. McIntosh and Kamei manually designed a set of code features, using six families of code change properties, which were primarily derived from prior studies~\cite{Kamei:2013:LES, Kim:2008:CSC, Kononenko:2015, Mockus2000}. These properties were: the magnitude of changes, the dispersion of the changes, the defect proneness of prior changes, the experience of the author, the code reviewers, and the degree of participation in the code review. Table~\ref{tab:metrics} briefly summarizes the code features extracted from code change properties.

\item DBNJIT: This approach adopted Deep Belief Network (DBN)~\cite{hinton2006reducing} to generate a more expressive set of features from an initial feature set~\cite{Yang:2015:DLJ}. The generated feature set, which is a nonlinear combination of the initial features, was put into a machine learning classifier~\cite{nasrabadi2007pattern} to predict buggy commits. For a fair comparison, we used McIntosh and Kamei~\cite{mcintosh2018fix}'s features as the initial feature set for DBNJIT. 
\end{itemize}

For all the above-mentioned techniques, we employ the same parameters and settings as described in the respective papers. 

\subsection{Evaluation Metric}
\label{sec:metric}
To evaluate the accuracy of \emph{Just-In-Time} (JIT) models, we calculate  threshold-independent measures of model performance. Since our dataset is imbalanced, we avoid using threshold-dependent measures (i.e., precision, recall, or F1) since these measures strongly depend on arbitraily thresholds~\cite{nguyen2009learning, gu2008data}. Following the previous work by McIntosh and Kamei~\cite{mcintosh2018fix}, we use the Area Under the receiver operator characteristics
Curve (AUC) to measure the discriminatory power of DeepJIT, i.e., their ability to differentiate between defective or clean commits. AUC computes the area under the curve plotting the true positive rate against the false positive rate, while applying multiple thresholds to determine if a commit is buggy or not. The values of AUC range between 0 (worst discrimination) and 1 (perfect discrimination).

\subsection{Training and hyperparameters}
\label{sec:training_parameters}
\begin{figure}[t!]
    \center
    \includegraphics[width=\linewidth]{figs/QT.pdf}
    \caption{The AUC results of DeepJIT across two different hyperparameters in QT project.}
    \label{fig:qt}
\end{figure}
\begin{figure}[t!]
    \center
    \includegraphics[width=\linewidth]{figs/OPENSTACK.pdf}
    \caption{The AUC results of DeepJIT across two different hyperparameters in OPENSTACK project.}
    \label{fig:openstack}
\end{figure}

One of the key challenges in training DeepJIT is how to select the dimension of the word vectors for the commit message ($d_m$) and code changes ($d_c$), and the size of the convolution layers (i.e., see Section~\ref{sec:cnn_msg} and Section~\ref{sec:cnn_code}). We evaluated the performance of DeepJIT, using \textit{5}-fold cross validation, across different word dimensions and number of filters. Figure~\ref{fig:qt} and Figure~\ref{fig:openstack} present the AUC results of DeepJIT for these hyperparameters. The figures show that DeepJIT achieves the best AUC results when the dimension of word vectors and the number of filters are set to 64. We set the other hyperparameters as follows: The batch size was set to 32. The size of DeepJIT's fully-connected layer described in Section~\ref{sec:ftr_combine} was set to 512. These hyperparameter settings are commonly used in prior deep learning work~\cite{severyn2015learning, huo2016learning, huo2017enhancing, hinton2012improving}.

We trained DeepJIT using Adam method~\cite{kingma2014adam} with shuffled mini-batches. We also trained DeepJIT for 100 epochs. We applied an early stopping strategy~\cite{prechelt1998automatic, caruana2001overfitting} to avoid overfitting problem during the training process. We stopped the training if the value of the objective function (see Equation~\ref{eq:cost}) has not been updated in the last 5 epochs. 

% For the size of the convolutional filters, we choose 64. The size of DeepJIT's fully-connected layer described in Section~\ref{sec:ftr_combine} is set to 512. The word vectors dimension of the commit message ($d_m$) and code changes ($d_c$) are set to 64. We train DeepJIT using Adam~\cite{kingma2014adam} with shuffled mini-batches.  The batch size is set to 32. We train DeepJIT for 100 epochs. We also apply the early stopping strategy~\cite{prechelt1998automatic, caruana2001overfitting} to avoid overfitting problem during the training process. Typically, we stop the training if the value of the objective function (see Equation~\ref{eq:cost}) has not been update in the last 5 epochs. All these hyperparameters in our paper are widely used in the deep learning community~\cite{severyn2015learning, huo2016learning, huo2017enhancing, hinton2012improving}. 
 
 

\subsection{Research Questions and Results}
\label{sec:rq_results}

\noindent \textbf{RQ1: How effective is DeepJIT compared to the state-of-the-art baseline?}

\begin{figure}
	\center
	\includegraphics[scale=0.4]{figs/split.pdf}
	\caption{An example of choosing the training data for short-period and long-period models. The last period is used as testing data.}
	\label{fig:splitting}
\end{figure}

\begin{table}[ht]
  \centering
  \caption{The AUC results of DeepJIT vs. with other baselines in three types of JIT models: cross-validation, short-period, and long-period.}
    \begin{tabular}{|c|l|c|c|}
    \hline
    \textbf{Settings} & \multicolumn{1}{c|}{\textbf{Models}} & QT & OPENSTACK \\
    \hline
    \hline
    \multirow{3}[6]{*}{Cross-validation} & JIT   & 0.701 & 0.691 \\
\cline{2-4}          & DBNJIT & 0.705 & 0.694 \\
\cline{2-4}          & DeepJIT & \textbf{0.768} & \textbf{0.751} \\
    \hline
    \multirow{3}[6]{*}{Short-Period} & JIT   & 0.703 & 0.711 \\
\cline{2-4}          & DBNJIT & 0.714 & 0.716 \\
\cline{2-4}          & DeepJIT & \textbf{0.764} & \textbf{0.781} \\
    \hline
    \multirow{3}[6]{*}{Long-period} & JIT   & 0.702 & 0.706 \\
\cline{2-4}          & DBNJIT & 0.708 & 0.712 \\
\cline{2-4}          & DeepJIT & \textbf{0.765} & \textbf{0.771} \\
    \hline
    \end{tabular}%
  \label{tab:results}%
\end{table}%

To address RQ1, we evaluated the accuracy of a trained JIT model in predicting buggy changes using test data. In particular, we considered three evaluation settings: 
\begin{itemize}
\item \textbf{Cross-validation:} To evaluate machine learning algorithm, most people use $k$-fold cross-validation~\cite{kohavi1995study} in which a dataset is randomly divided to $k$ folds, each fold is considered as testing data for evaluating JIT model while $k - 1$ folds are considered as training data. In this case, the JIT model is trained on a mixture of past and future data. In our paper, we set $k = 5$.
\item \textbf{Short-period:} The JIT model is trained using commits that occurred at one time period. We assume that older commits changes may have characteristics that no longer effects to the latest commits. 
\item \textbf{Long-period:} Inspired by the work~\cite{rahman2013sample}, suggesting that larger amounts of training data tend to achieve a better performance in defect prediction problem, we train the JIT model using all commits that occurred before a particular period. We discover whether additional data may improve the performance of the JIT model. 
\end{itemize} 

Figure~\ref{fig:splitting} describes how the training data is selected to train models  following the short-period and long-period settings. We used the last period (i.e., period 5) as a testing data. While the short-period model was trained using the commits that occurred during period 4, the long-period model was trained using the commits that occurred from period 1 to 4. After training short-period and long-period models, we measured their performance using AUC evaluation metric described in Section~\ref{sec:metric}.

Table~\ref{tab:results} shows the AUC results of DeepJIT as well as other baselines considering the three evaluation settings: cross-validation, short-period, and long-period. The difference between results obtained using cross-validation, short-period, and long-period settings is relatively small (i.e., below 2.2\%) which suggests that there is no difference between training on past or future data. 
% \cmt{TODO: Prof. Hoa, do you have any explaination about it?} 
In the QT project, DeepJIT achieved AUC scores of 0.768, 0.764, and 0.765 in three different evaluation settings: cross-validation, short-period, and long-period, respectively. Comparing them to the best performing baseline (i.e., DBNJIT), DeepJIT achieved improvements of 8.96\%, 7.00\%, and 8.05\% in terms of AUC. In the OPENSTACK project, DeepJIT also constituted improvements of 8.21\%, 9.08\%, and 8.29\% in terms of AUC compared to DBNJIT (the best performing baseline). We also employed the Scott-Knott test~\cite{ghotra2015revisiting} on the cross-validation evaluation setting to statistically compare the differences between the three considered JIT models. The results show that DeepJIT consistently appears in the top Scott-Knott ESD rank in terms of AUC (i.e, DeepJIT $>$ DBNJIT $>$ JIT).  

\hspace{0.5cm}

\noindent \textbf{RQ2: Does the proposed model benefit from both commit message and the code changes?}

\begin{table}[ht!]
  \centering
  \caption{Contribution of feature components in DeepJIT.}
    \begin{tabular}{|c|l|c|c|}
    \hline
    \textbf{Settings} & \multicolumn{1}{c|}{\textbf{Models}} & QT & OPENSTACK \\
    \hline
    \hline
    \multirow{3}[6]{*}{Cross-validation} & DeepJIT-Msg & 0.641 & 0.689 \\
\cline{2-4}          & DeepJIT-Code & 0.738 & 0.729 \\
\cline{2-4}          & DeepJIT & \textbf{0.768} & \textbf{0.751} \\
    \hline
    \multirow{3}[6]{*}{Short-Period} & DeepJIT-Msg & 0.609 & 0.583 \\
\cline{2-4}          & DeepJIT-Code & 0.734 & 0.769 \\
\cline{2-4}          & DeepJIT & \textbf{0.764} & \textbf{0.781} \\
    \hline
    \multirow{3}[6]{*}{Long-period} & DeepJIT-Msg & 0.638 & 0.659 \\
\cline{2-4}          & DeepJIT-Code & 0.727 & 0.738 \\
\cline{2-4}          & DeepJIT & \textbf{0.765} & \textbf{0.771} \\
    \hline
    \end{tabular}%
  \label{tab:variants}%
\end{table}%

% \begin{table*}[t!]
%   \centering
%   \caption{Contribution of feature components in DeepJIT.}
%     \begin{tabular}{|l|c|c|c|c|c|c|}
%     \hline
%     \multirow{2}[4]{*}{} & \multicolumn{3}{c|}{QT} & \multicolumn{3}{c|}{OPENSTACK} \\
% \cline{2-7}          & \multicolumn{1}{l|}{\textbf{Cross-validation}} & \multicolumn{1}{l|}{\textbf{Short-period}} & \multicolumn{1}{l|}{\textbf{Long-period}} & \multicolumn{1}{l|}{\textbf{Cross-validation}} & \multicolumn{1}{l|}{\textbf{Short-period}} & \multicolumn{1}{l|}{\textbf{Long-period}} \\
%     \hline
%     \hline
%     DeepJIT-Msg & 0.641 & 0.609 & 0.638 & 0.689 & 0.583 & 0.659 \\
%     \hline
%     DeepJIT-Code & 0.738 & 0.734 & 0.727 & 0.729 & 0.769 & 0.738 \\
%     \hline
%     DeepJIT & \textbf{0.768} & \textbf{0.764} & \textbf{0.765} & \textbf{0.751} & \textbf{0.781} & \textbf{0.771} \\
%     \hline
%     \end{tabular}%
%   \label{tab:variants}%

% \end{table*}%

To answer this question, we employed an ablation test~\cite{korbar2017deep, liu2017deep}, by ignoring the commit message and the code change in a commit and then evaluate the AUC performance. Specifically, we created two different variants of DeepJIT, namely DeepJIT-Msg and DeepJIT-Code. DeepJIT-Msg only considers commit message information while DeepJIT-Code only uses commit code information. We again used the three evaluation settings (i.e., cross-validation, short-period, and long-period) and the AUC scores to evaluate the performance of our models. Table~\ref{tab:variants} shows the performance of DeepJIT degrades if we ignore any one of the considered types of information (i.e. commit messages or code changes). The AUC scores dropped by 19.81\%, 28.45\%, and 19.01\% in the project QT and dropped by 9.00\%, 33.96\%, and 16.00\% in the project OPENSTACK for the three evaluation settings if we ignore commit messages. The AUC scores dropped by 4.07\%, 4.09\%, and 5.23\% in the project QT and dropped by 3.02\%, 1.56\%, and 4.47\% in the project OPENSTACK for the three evaluation settings if we ignore code changes information. It suggests that each information type contributes to DeepJIT's performance. Moreover, it also indicates that code changes are more important to detect buggy commits than commit messages. 

\hspace{0.5cm}

\noindent \textbf{RQ3: Does DeepJIT benefit from the manually extracted code changes features?}

\begin{table}[ht]
  \centering
  \caption{Combination of DeepJIT with the manually crafted code features from~\cite{mcintosh2018fix}.}
    \begin{tabular}{|c|l|c|c|}
    \hline
    \textbf{Settings} & \multicolumn{1}{c|}{\textbf{Models}} & QT & OPENSTACK \\
    \hline
    \hline
    \multirow{2}[4]{*}{Cross-validation} & DeepJIT & 0.768 & 0.751 \\
\cline{2-4}          & DeepJIT-Combined & \textbf{0.779} & \textbf{0.76} \\
    \hline
    \multirow{2}[4]{*}{Short-Period} & DeepJIT & 0.764 & 0.781 \\
\cline{2-4}          & DeepJIT-Combined & \textbf{0.788} & \textbf{0.814} \\
    \hline
    \multirow{2}[4]{*}{Long-period} & DeepJIT & 0.765 & 0.771 \\
\cline{2-4}          & DeepJIT-Combined & \textbf{0.786} & \textbf{0.799} \\
    \hline
    \end{tabular}%
  \label{tab:combined}%
\end{table}%

To address this question, we incorporated the code features, derived from~\cite{mcintosh2018fix}, into our proposed model. Specifically, the code features, namely $\textbf{z}_\textbf{r}$, are concatenated with the two embedding vectors  $\textbf{z}_\textbf{m}$ and $\textbf{z}_C$, representing the salient features of commit message and code change (see Section~\ref{sec:ftr_combine}), to build a new single vector $\textbf{z}$ as follows:
\begin{equation}
\label{eq:combined_ftr}
\textbf{z} = \textbf{z}_\textbf{m} \oplus \textbf{z}_C \oplus \textbf{z}_\textbf{r}
\end{equation}
where $\oplus$ is the concatenation operator. Table~\ref{tab:combined} shows the AUC results of a DeepJIT variant (referred to as DeepJIT-Combined) that also leverages McIntosh and Kamei~\cite{mcintosh2018fix}'s manually crafted features. We find that the AUC scores increased by 1.43\%, 3.14\%, and 2.75\% in the project QT and they increased by 1.20\%, 4.23\%, and 3.63\% in the project OPENSTACK for the three evaluation settings (i.e. cross-validation, short-period, long-period). DeepJIT-Combined improved the best baseline model (i.e. DBNJIT) by 10.50\%, 10.36\%, and 11.02\% in the project QT and 9.51\%, 13.69\%, 12.22\% in the project OPENSTACK for the there evaluation settings. This suggests that the manually extracted code features are complementary and can be used to improve the performance of our proposed approach.

\hspace{0.5cm}

\noindent \textbf{RQ4: What are the time costs of DeepJIT?}
% \begin{table*}[t!]
%   \centering
%   \caption{Time costs of DeepJIT.}
%     \begin{tabular}{|c|c|c|c|c|c|c|}
%     \hline
%     \multicolumn{1}{|c|}{\multirow{2}[4]{*}{Dataset}} & \multicolumn{2}{c|}{\textbf{Cross-validation}} & \multicolumn{2}{c|}{\textbf{Short-period}} & \multicolumn{2}{c|}{\textbf{Long-period}} \\
% \cline{2-7}          & Training time & Testing time & Training time & Testing time & Training time & Testing time \\
%     \hline
%     \hline
%     QT    & 5 hours 43 mins & 36.2 mins & 17.2 mins & 3.2 mins & 1 hours 18 mins & 8.1 mins \\
%     \hline
%     OPENSTACK & 12 hours 15 mins & 1 hours 6 mins & 10.1 mins & 2.3 mins & 2 hours 37 mins & 12.4 mins \\
%     \hline
%     \end{tabular}%
%   \label{tab:cost}%
% \end{table*}%

\begin{table}[t!]
  \centering
  \caption{Training time of DeepJIT}
    \begin{tabular}{|l|c|c|c|}
    \hline
    \multicolumn{1}{|c|}{Dataset} & Cross-validation & Short-period & Long-period \\
    \hline
    \hline
    QT    & 5 hours 43 mins & 17.2 mins & 1 hours 18 mins \\
    \hline
    OPENSTACK & 12 hours 15 mins & 10.1 mins & 2 hours 37 mins \\
    \hline
    \end{tabular}%
  \label{tab:cost}%
\end{table}%

We trained and tested DeepJIT on a NVIDIA DGX1 server with Tesla P100~\cite{gawande2018scaling}. Table~\ref{tab:cost} shows the time costs of training DeepJIT for the three evaluation settings (i.e., cross-validation, short-period, and long-period) on QT and OPENSTACK. Cross-validation setting requires longest training time since we performed $5$-fold cross-validation to evaluate the performance of DeepJIT. Long-period setting requires more training time than short-period setting since it considers all commits occurring before a particular period. Once DeepJIT has been trained, it only takes a few milliseconds to generate the prediction score for a given commit.
% \cmt{Prof. Hoa: do you have any idea how to describe the time cost?}


\section{Threats to Validity}
\label{sec:threat}
We mitigated concerns related to construct validity  by evaluating our approach on a publicly available dataset that has been used in previous work. This dataset contains commits extracted from real projects (QT and OPENSTACK) and buggy/no-buggy labels on those commits. Threats to conclusion validity was also minimized by using Area Under the Curve (AUC), a standard performance measure which is recommended for assessing the predictive performance of defect prediction models \cite{tantithamthavorn2018optimization}. We however acknowledge that the Statistical tests such as Scott-Knott ESD rank with effect size or Mann-Whitney's U test can be used to confirm the statistical significance of our conclusions and plan to investigate this in our future work.

We have compared our approach against two baselines which have been proposed and implemented in existing work. Since the source code of their original implementation were not made publicly available, we needed to re-implement our own version of those techniques. Our implementation closely follows the description of their work, it might not have all of the details of the original implementation, specifically those not explicitly presented in their papers. Our study considers two large open source projects which are significantly different in size, complexity and revision history. However, due to small sample sizes, our findings may not generalize to all software projects. Further studies are needed to confirm our results for other types of software projects.
\section{Related Work}
\label{sec:related}
\subsection{JIT Defect Prediction}
Some previous studies focus on change-level defect prediction (i.e., JIT defect prediction). For example, Mockus and Weiss~\cite{Mockus2000} predict commits as being buggy or not in an industrial project. They use metric-based features, such as the number of subsystems touched, the number of files modified, the number of lines of added code, and the number of modification requests. Motivated by their previous work, Kamei et al.~\cite{Kamei:2013:LES} built upon the set of code change features, reporting that the addition of a variety of features that were extracted from the Version Control System (VCS) and the Issue Tracking System (ITS) helped to improve the prediction accuracy. They conduct an empirical study of the effectiveness of JIT defect prediction on a set of six open source and five commercial projects and also evaluate their findings when considering the effort required to review the changes.

Aversano \emph{et al.}~\cite{Aversano2007} and Kim \emph{et al.}~\cite{Kim2008} use source code change logs to predict commits as being buggy or not. For example, Kim \emph{et al.}~\cite{Kim2008} used the identifiers in added and deleted source code and the words in change logs. The experimental results on the datasets collected from 12 open source software projects show that the proposed approach achieve 78 percent accuracy and a 60 percent recall.

Kononenko et al.~\cite{Kononenko:2015} found that the addition of code change properties that were extracted from code review databases contributed a significant amount of explanatory power to JIT models. McIntosh and Kamei also 
\cmt{TODO: update here} 
%The importance of impactful families of code change properties like Size and Review are consistently under/overestimated in the studied systems.

Comparing with these previous studies, we introduce the JIT defect prediction model (DeepJIT) that learn a deep representation of commits and compare the prediction performance of DeepJIT with other JIT models on the dataset including code change properties that are extracted from code review databases. We extended the datasets that McIntosh and Kamei used to analyze~\footnote{\url{https://github.com/software-rebels/JITMovingTarget}} by adding commit messages and code changes. 

\subsection{Deep Learning Models in Defect Prediction}

%Xin Xia's group

%Lin Tang's group

%Say the difference between them and us.
\section{Conclusion and Future Work}
\label{sec:conclusion}
In this paper, we propose an end-to-end deep learning model (namely DeepJIT) for \emph{Just-In-Time} defect prediction problem. For a given commit, DeepJIT automatically extracts features from the commit message and the set of code changes. These features are then combined to evaluate how likely the commit is buggy. DeepJIT also allows users to add their manually crafted features to make it more robust. We evaluate DeepJIT on two popular software projects (i.e. QT and OPENSTACK) on three evaluation settings (i.e. cross-validation, short-period, and long-period). The evaluation results show that compared to the best performing state-of-the-art baseline (DBNJIT), the best variant of DeepJIT (DeepJIT-Combined) achieves improvements of 10.50\%, 10.36\%, and 11.02\% in the project QT and 9.51\%, 13.69\%, 12.22\% in the project OPENSTACK in terms of the Area Under the Curve (AUC).

Our future work involves extending our evaluation to other open source and commercial projects. We also plan to extend DeepJIT using attention neural network~\cite{yin2016abcnn} so that our model can explain its predictions to software practitioners. We also plan to implement DeepJIT into a tool (e.g. a Github plugin) to assess its usefulness in practice.

\noindent \textbf{Dataset and Code.} The dataset and code for DeepJIT are available at~\url{https://github.com/AnonymousAccountConf/DeepJTT_MSR}.

\noindent \textbf{Acknowledgements.} This research was partially supported by JSPS KAKENHI Grant Numbers JP15H05306 and JP18H03222.

%, e.g., identification of valid commits in automated program repair~\cite{xiong2018identifying}, assignment of commits to developers for code review~\cite{thongtanunam2015should, zanjani2016automatically}

\bibliographystyle{IEEEtran}
\bibliography{bib,JIT,TSE_ChangeRisk}

\end{document}
