\section{Related Work}
\label{sec:related}
\subsection{JIT Defect Prediction}
Some previous studies focus on change-level defect prediction (i.e., JIT defect prediction). For example, Mockus and Weiss~\cite{Mockus2000} predict commits as being buggy or not in an industrial project. They use metric-based features, such as the number of subsystems touched, the number of files modified, the number of lines of added code, and the number of modification requests. Motivated by their previous work, Kamei et al.~\cite{Kamei:2013:LES} built upon the set of code change features, reporting that the addition of a variety of features that were extracted from the Version Control System (VCS) and the Issue Tracking System (ITS) helped to improve the prediction accuracy. They conduct an empirical study of the effectiveness of JIT defect prediction on a set of six open source and five commercial projects and also evaluate their findings when considering the effort required to review the changes.

Aversano \emph{et al.}~\cite{Aversano2007} and Kim \emph{et al.}~\cite{Kim2008} use source code change logs to predict commits as being buggy or not. For example, Kim \emph{et al.}~\cite{Kim2008} used the identifiers in added and deleted source code and the words in change logs. The experimental results on the datasets collected from 12 open source software projects show that the proposed approach achieve 78 percent accuracy and a 60 percent recall.

Kononenko et al.~\cite{Kononenko:2015} found that the addition of code change properties that were extracted from code review databases contributed a significant amount of explanatory power to JIT models. McIntosh and Kamei also 
\cmt{TODO: update here} 
%The importance of impactful families of code change properties like Size and Review are consistently under/overestimated in the studied systems.

Comparing with these previous studies, we introduce the JIT defect prediction model (DeepJIT) that learn a deep representation of commits and compare the prediction performance of DeepJIT with other JIT models on the dataset including code change properties that are extracted from code review databases. We extended the datasets that McIntosh and Kamei used to analyze~\footnote{\url{https://github.com/software-rebels/JITMovingTarget}} by adding commit messages and code changes. 

\subsection{Deep Learning Models in Defect Prediction}

%Xin Xia's group

%Lin Tang's group

%Say the difference between them and us.