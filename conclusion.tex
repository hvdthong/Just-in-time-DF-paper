\section{Conclusion and Future Work}
\label{sec:conclusion}
In this paper, we propose an end-to-end deep learning model (namely DeepJIT) for \emph{Just-In-Time} defect prediction problem. For a given commit, DeepJIT automatically extract features from the commit message and the set of code changes. These features are then combined to evaluate how likely the commit is buggy. DeepJIT also allows users to add their manually features to make it more powerful. We evaluate DeepJIT on two popular software projects (i.e., QT and OPENSTACK) on three different settings (i.e., cross-validation, short-period, and long-period) shows that DeepJIT achieves improvements of 10.50\%, 10.36\%, and 11.02\% in the project QT and 9.51\%, 13.69\%, 12.22\% in the project OPENSTACK in terms  of the Area Under the Curve (AUC).

In the future work, we want to extend our the dataset to cover more applications, e.g., identification of valid commits in automated program repair~\cite{xiong2018identifying}, assignment of commits to developers for code review~\cite{thongtanunam2015should, zanjani2016automatically}. We also develop explainability features into to DeepJIT using attention neural network~\cite{yin2016abcnn}. Finally, we will implement DeepJIT into a tool (e.g., Github plugin).

\noindent \textbf{Dataset and Code.} The dataset and code for DeepJIT are available at~\url{https://github.com/AnonymousAccountConf/DeepJTT_MSR}.