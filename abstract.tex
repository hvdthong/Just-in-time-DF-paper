\begin{abstract}
Most software Quality Assurance (SOA) resources often focus on software modules that are likely to be defective in the future to help developers saving their effort to debug a program. To solve this problem, change-level defect prediction or Just-in-time (JIT) defect prediction is proposed to identify bug in the code changes. JIT models are trained using machine learning techniques which assume that historical changes are similar to future one. Hence, these changes can be used to identify defect-prone software modules (e.g., functions, files, system, etc.). A previous approach relies on manually extracted code changes features. This approach, however, shows only moderate accuracy. In this paper, we propose a novel deep learning framework that is automatically extracting features from commit message and code changes and using them to identify bugs. Our framework takes into account the hierarchical structure of code changes to produce their features. Experiments on two well-known projects (i.e., QT and OPENSTACK) shows that our proposed approach outperforms the state-of-the-art baseline in term of the area under the receiver operator characteristics Curve (AUC). 
\end{abstract}