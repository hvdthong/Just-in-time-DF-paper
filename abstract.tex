\begin{abstract}
Software Quality Assurance efforts often focus on identifying software modules that are likely to be defective in the future. To solve the problem of finding likely defective code early, change-level defect prediction -- aka. \emph{Just-In-Time} (JIT) defect prediction -- has been proposed to identify buggy code changes. JIT models are trained using machine learning techniques which assume that historical changes are similar to future ones. Hence, these changes can be used to identify defect-prone software modules (e.g., functions, files, system, etc.). A previous approach relies on manually extracted code changes features. This approach, however, shows only moderate accuracy. In this paper, we propose an end-to-end deep learning framework, namely DeepJIT, that is automatically extracting features from commit messages and code changes and using them to identify bugs. Experiments on two popular software projects (i.e., QT and OPENSTACK) on three evaluation settings (i.e., cross-validation, short-period, and long-period) show that DeepJIT achieves improvements of 10.50\%, 10.36\%, and 11.02\% in the project QT and 9.51\%, 13.69\%, 12.22\% in the project OPENSTACK in terms of the Area Under the Curve (AUC). 
\end{abstract}