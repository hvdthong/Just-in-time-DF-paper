\begin{abstract}
Software quality assurance efforts often focus on identifying defective code. To find likely defective code early, change-level defect prediction -- aka. \emph{Just-In-Time} (JIT) defect prediction -- has been proposed. JIT defect prediction models identify likely defective changes and they are trained using machine learning techniques with the assumption that historical changes are similar to future ones. Most existing JIT defect prediction approaches make use of manually engineered features. Unlike those approaches, in this paper, we propose an end-to-end deep learning framework, named DeepJIT, that automatically extracts features from commit messages and code changes and use them to identify defects. Experiments on two popular software projects (i.e., QT and OPENSTACK) on three evaluation settings (i.e., cross-validation, short-period, and long-period) show that the best variant of DeepJIT (DeepJIT-Combined), compared with the best performing state-of-the-art approach, achieves improvements of 10.36-11.02\% for the project QT and 9.51-13.69\% for the project OPENSTACK in terms of the Area Under the Curve (AUC). 
\end{abstract}